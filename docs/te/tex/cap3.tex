%%%%%%%%%%%%%%%%%%%%%%%%%%%%%%%%%%%%%%%%%%%%%%%%%%%%%%%%%%%%%%%%%%%%%%%%%%%%%%%
% Chapter 3: Procedimiento experimental 
%%%%%%%%%%%%%%%%%%%%%%%%%%%%%%%%%%%%%%%%%%%%%%%%%%%%%%%%%%%%%%%%%%%%%%%%%%%%%%%

%++++++++++++++++++++++++++++++++++++++++++++++++++++++++++++++++++++++++++++++
\section{Descripci�n de los experimentos}
\label{3:sec:1}

  El experimento llevado a cabo en esta memoria ha consistido en la realizaci�n de varios c�digos en lenguaje $Python$. Los algoritmos implementados que solucionan dichos c�digos 
estiman la aproximaci�n $f(x) = sin(x)$ mediante el m�todo de Taylor, solicitando el grado del polinomio de Taylor, el punto central y el punto x donde se evalua dicho polinomio.  
  

%++++++++++++++++++++++++++++++++++++++++++++++++++++++++++++++++++++++++++++++
\section{Descripci�n del material}
\label{3:sec:2}
  Los materiales requeridos para la realizaci�n del trabajo han sido: 
\begin{itemize}
 
  \item
   CPU type: Intel(R) Core(TM) i3-2328M CPU @ 2.50GHz
  \item
   vendor ID	GenuineIntel
  \item
    CPU speed	1200.000Hz
  \item
    cache size	2048 KB
  
\end{itemize}



%++++++++++++++++++++++++++++++++++++++++++++++++++++++++++++++++++++++++++++++
\section{Resultados obtenidos}
\label{3:sec:3}




%------------------------------------------------------------------------------
\begin{figure}[!th]
\begin{center}
\includegraphics[width=0.75\textwidth]{images/figura1.eps}
\caption{Ejemplo de figura}
\label{fig:1}
\end{center}
\end{figure}
%------------------------------------------------------------------------------


%------------------------------------------------------------------------------
\begin{table}[!ht]

\begin{tabular}{|l|c|c|c|c|c|}
\hline
Grado  & Punto c & Punto x & Aproximacion       & Error             & Tiempo CPU           \\ \hline
4      &  10.0   &-10.0    & -4620.52781304922  & 4619.98379193833  & 0.00790095329284668 \\ \hline
4      &  10.0   & -5.0    & -1545.27783908746  & 1545.73381797657  & 0.00798487663269043  \\ \hline
4      &  10.0   &  5.0    & -21.1962728645601  & 20.6522517536707  & 0.010929107666015625 \\ \hline
4      &  10.0   &  9.0    & 0.404548172498635  &-0.948569283388005 & 0.008239030838012695 \\ \hline
4      &  10.0   &  9.9    & -0.457535964436753 &-0.086485146452617 & 0.00795292854309082  \\ \hline
4      &  10.0   & 10.0    & -0.544021110889370 &       0           & 0.008179903030395508  \\ \hline
\end{tabular}

\caption{Tabla de datos obtenidos. Experimento 1}
\label{tab}
\end{table}

\begin{table}[!ht]

\begin{tabular}{|l|c|c|c|c|c|}
\hline
Grado  & Punto c & Punto x & Aproximacion       & Error             & Tiempo CPU           \\ \hline
4      & -7.0    &  2.0    & -238.466744388979  & 237.809757790260  & 0.008273124694824219 \\ \hline
4      & -2.0    &  2.0    & -0.559778321379882 &-0.349519105445799 & 0.008683919906616211 \\ \hline
4      &  0.0    &  2.0    & 0.666666666666667  &-0.666666666666667 & 0.007463932037353516 \\ \hline
4      &  2.0    &  2.0    & 0.909297426825682  &        0          & 0.007899999618530273  \\ \hline
4      &  7.0    &  2.0    & 21.4904658168047   & -20.8334792180859 & 0.007979869842529297  \\ \hline
\end{tabular}

\caption{Tabla de datos obtenidos. Experimento 2}
\label{tab}
\end{table}

\begin{table}[!ht]

\begin{tabular}{|l|c|c|c|c|c|}
\hline
Grado  & Punto c & Punto x & Aproximacion       & Error             & Tiempo CPU            \\ \hline
10     & 10.0    & -10.0   & 2226636.13194962   & -2226636.67597073 & 0.012552976608276367  \\ \hline
10     & 10.0    &  -5.0   & 124685.902313545   & -124686.446334656 & 0.011905908584594727  \\ \hline
10     & 10.0    &   0.0   & 1920.68755204547   & -1921.23157315635 & 0.011946916580200195  \\ \hline
10     & 10.0    &   5.0   & 0.163743739637233  & 0.707764850526603 & 0.08893513679504395   \\ \hline
10     & 10.0    &   9.0   & 0.412118507257685  &-0.956139618147055 & 0.014183998107910156  \\ \hline
10     & 10.0    &   9.9   & -0.457535893775321 &0.0864852171140484 & 0.012012958526611328  \\ \hline
10     & 10.0    &  10.0   & -0.544021110889370 &        0          & 0.014500856399536133  \\ \hline
\end{tabular}

\caption{Tabla de datos obtenidos. Experimento 3}
\label{tab}
\end{table}


\begin{table}[!ht]

\begin{tabular}{|l|c|c|c|c|c|}
\hline
Grado  & Punto c & Punto x & Aproximacion       & Error             & Tiempo CPU            \\ \hline
100    & -1.0    &-0.99999 &-0.8414655817427645 &-5.40306513208133e & 0.06290507316589355   \\ \hline
100    & -1.0    & -0.9    & -0.783326909627484 &-0.0581440751804130& 0.06340909004211426   \\ \hline

\end{tabular}

\caption{Tabla de datos obtenidos. Experimento 4}
\label{tab}
\end{table}


%------------------------------------------------------------------------------

%++++++++++++++++++++++++++++++++++++++++++++++++++++++++++++++++++++++++++++++
\section{An�lisis de los resultados}
\label{3:sec:4}

bla, bla, etc. 

