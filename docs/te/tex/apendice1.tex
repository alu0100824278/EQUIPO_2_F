\section{Algoritmo principal}
\label{Apendice1:XXX}

\begin{center}
\begin{footnotesize}
\begin{verbatim}

PROGRAMA PRINCIPAL

#!/src/bin/python
#!encoding: UTF-8
import math 
from sympy import *
import modulo
import time
import numpy as np 
import matplotlib.pyplot as mp

numero = int(raw_input("Introduzca el grado que desea que tenga el Polinomio de Taylor: "))
centro = float(raw_input("Introduzca el punto central donde desea que se evalue el Polinomio de Taylor: "))
x = float(raw_input("Introduzca el punto x donde desea evaluar el Polinomio de Taylor: "))

modulo.taylor1(x, numero, centro)
y1 = np.linspace(-np.pi, np.pi, 256, endpoint=True)
y1 = np.arange(-8,8,0.1)
y2 = np.sin(y1)
  
mp.plot(y1,y2, color = 'blue',linewidth=2.5, linestyle="-", label="seno")  
mp.legend(loc=0) 

# Para poner el simbolo pi en el eje x se usa LaTeX.
mp.xticks([-2*np.pi, -3*np.pi/2, -np.pi, -np.pi/2, 0, np.pi/2, np.pi, 3*np.pi/2, 2*np.pi],
[r'$2\pi$',r'$-\frac{3\pi}{2}$',r'$-\pi$',r'$-\frac{\pi}{2}$',r'$0$',r'$+\frac{\pi}{2}$',r'$+\pi$',
r'$\frac{3\pi}{2}$',r'$2\pi$'])  

mp.title("Representacion grafica")         
 
mp.savefig('grafico.png',dpi = 100)

mp.show()

\end{verbatim}
\end{footnotesize}
\end{center}

\section{Algoritmo del m�dulo}
\label{Apendice1:YYY}

\begin{center}
\begin{footnotesize}
\begin{verbatim}
/#!/src/bin/python

import math
from sympy import *
import time

def factorial(numero):
  factorial = 1
  while(numero > 1):
    factorial = factorial * numero
    numero = numero - 1
  return factorial
  
\end{verbatim}
\end{footnotesize}
\end{center}
  
\begin{center}
\begin{footnotesize}
\begin{verbatim}

def taylor1(x, numero, centro):
  inicio = time.time()
  c = Symbol('c')
  f = sin(c)
  func = f.evalf(subs={c: centro})
  suma = func
  for i in range (1,numero + 1):
    f_i = diff(f, c)
    v = f_i.evalf(subs={c: centro})
    s= (v / factorial(i)) * ((x - centro) ** i)    
    suma = suma + s
    f = f_i
  error = func - suma
  print 'El valor de la funcion original sin(%f) es igual a %f ' % (centro, func)
  print 'El Polinomio de Taylor de grado n=%d en el punto centro c=%f 
  evaluada en el punto x=%f es igual a %f' % (numero, centro, x, suma)
  print 'El Error de la funcion original con el Polinomio de Taylor es: error=%f' % error
  l=[numero, centro, x, suma, error]
  f = open("Taylor.tex", 'a')
  
\end{verbatim}
\end{footnotesize}
\end{center}

\begin{center}
\begin{footnotesize}
\begin{verbatim}
  f.write('Grado del Polinomio (n) Punto Central (c)  Punto de Evaluacion (x)  
  Aproximacion
  Error Tiempo CPU \n ')
  fin = time.time()
  tiempo_total = fin - inicio
  l=l+[tiempo_total]
  f.write(str(l))
  f.write("\n")
  f.close()
  f=open("Taylor.tex","r")
  print(f.read())
  f.close()
  return taylor1
  
\end{verbatim}
\end{footnotesize}
\end{center}
