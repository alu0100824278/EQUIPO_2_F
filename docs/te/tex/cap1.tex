%%%%%%%%%%%%%%%%%%%%%%%%%%%%%%%%%%%%%%%%%%%%%%%%%%%%%%%%%%%%%%%%%%%%%%%%%%%%%
% Chapter 1: Motivaci�n y Objetivos 
%%%%%%%%%%%%%%%%%%%%%%%%%%%%%%%%%%%%%%%%%%%%%%%%%%%%%%%%%%%%%%%%%%%%%%%%%%%%%%%

  A lo largo de este curso hemos aprendido a implementar diferentes c�digos en $Python$, los cuales han logrado generar nuestra curiosidad por saber 
m�s. Esto nos permiti� ir m�s all� y poder fusionar dicho lengutaje de programaci�n con el procesador de texto $\LaTeX$ y una clase de este, el $Bearmer$, que utilizamos para realizar presentaciones. A partir de todos ellos, hemos conseguido llevar a cabo esta memoria. 
%---------------------------------------------------------------------------------
\section{Objetivo principal:}
\label{1:sec:1}
  Profundizar nuestros conocimientos con el lenguaje de programaci�n $Python$, el procesador de texto $\LaTeX$ y el creador de presentaciones $Bearmer$
sobre el estudio de las series de Taylor.

%---------------------------------------------------------------------------------
\section{Objetivo espec�fico:}
\label{1:sec:2}
  Hallar la aproximaci�n de $f(x) = sin(x)$ mediante el m�todo de Taylor, el error cometido y el estudio del tiempo de programaci�n.


