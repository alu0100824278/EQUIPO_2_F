\section{Explicacion del Algoritmo}
\label{Apendice2:label}

\begin{center}
\begin{footnotesize}
\begin{verbatim}
En el programa principal primero, importamos las 
librerias que usaremos mas adelante con el comando import o from xxx import yyy 
para importar un elemento concreto de la libreria.
\end{verbatim}
\end{footnotesize}
\end{center}

\section{Explicacion del algoritmo del modulo}
\label{Apendice2:label2}

\begin{center}
\begin{footnotesize}
\begin{verbatim}
Luego, pedimos por pantalla las 3 variables necesarias para ejecutar nuestro programa que son:
el grado del polinomio de Taylor (numero), el punto central (centro) 
y el punto donde se desea evaluar (x). 
Finalmente, acabamos el programa llamando a la funcion taylor1 
que se encuentra en nuestro modulo.

Al igual que en el programa principal, importaremos las librerias 
que utilizaremos posteriormente. 
Primeramente, definimos la funcion factorial, que luego sera usada 
en la otra funcion para calcular el polinomio de Taylor. factorial(numero) 
crea un bucle for que incremente su valor, 
dependiendo de la variable numero introducida.
 
\end{verbatim}
\end{footnotesize}
\end{center}

\begin{center}
\begin{footnotesize}
\begin{verbatim}
La siguiente funcion taylor1 dependera de las tres variables que se han 
introducido por pantalla.

Primero, declaramos una variable c sobre que la que se derivara luego la funcion f 
respecto de ella

Evaluamos la funcion f en el punto c = centro y con ese valor 
inicializamos la variable suma. Luego, entramos en un bucle for, 
en el que se vaya derivando la funcion f y asi, se vaya evaluando en el punto c = centro
para calcular una nueva variable s, propia del polinomio de Taylor,
y asi ir incrementando la variable suma.
 
Ahora calculamos el error cometido entre la funcion original 
y la aproximacion del polinomio de Taylor. 
Seguidamente, mostramos por pantalla los datos obtenidos en el programa 
y creamos una lista l que luego usaremos para escribir en un fichero.
 
Lo siguiente sera parar el cronometro del programa 
y calcular el tiempo total que ha tardado en realizarse.
 
Por ultimo, abrimos un fichero de nombre Taylor.tex 
que actualice lo que hemos obtenido como resultados, 
si no se encuentra ya en el fichero ("a"), 
luego escribimos un encabezado, para saber que significa cada dato 
que luego escribiremos en una única línea con f.write(str(l)).
 
Finalizamos el modulo con un f.close() para cerrar el fichero 
y mostramos por pantalla dicho fichero para comprobar los resultados.
 
\end{verbatim}
\end{footnotesize}
\end{center}
