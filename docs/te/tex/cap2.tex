%%%%%%%%%%%%%%%%%%%%%%%%%%%%%%%%%%%%%%%%%%%%%%%%%%%%%%%%%%%%%%%%%%%%%%%%%%%%%%%
% Chapter 2: Fundamentos Te�ricos 
%%%%%%%%%%%%%%%%%%%%%%%%%%%%%%%%%%%%%%%%%%%%%%%%%%%%%%%%%%%%%%%%%%%%%%%%%%%%%%%


\section{Historia}
\label{2:sec:1}
  Brook Taylor naci� el 18 de agosto de 1685 en Edmonton. Hijo de John Taylor,
del Parlamento de Bifrons, Kent, y de Olivia Tempest (hija de Sir Nicholas Tempest). 

  En "Los m�todos de incrementaci�n directa e inversa" de Taylor (1715) agregaba a las 
matem�ticas una nueva rama llamada ahora  <<El c�lculo de las diferencias finitas>> , 
e invent� la integraci�n por partes y descubri� la c�lebre f�rmula conocida como la Serie de Taylor,
la importancia de esta f�rmula no fue reconocida hasta 1772, cuando Lagrange proclam� los 
principios b�sicos del C�lculo Diferencial. Taylor tambi�n desarroll� los principios fundamentales 
de la perspectiva en "Perspectivas Lineales" (1715). En su Methodus Incrementorum Directa et Inversa 
(Londres, 1715) desarroll� una nueva parte dentro de la investigaci�n matem�tica, 
que hoy se llama c�lculo de las diferencias finitas. Junto con "Los nuevos principios de 
la perspectiva lineal". Taylor da cuenta de un experimento para descubrir las leyes de la 
atracci�n magn�tica (1715) y un m�todo no probado para aproximar las ra�ces de una ecuaci�n 
dando un m�todo nuevo para logaritmos computacionales (1717). 

  Brook Taylor muri� en Somerset House el 29 de diciembre de 1731.

\section{C�lculo de la serie de Taylor}
\label{2:sec:2}
  Sea $f(x)$ una funci�n definida en un intervalo que contiene al punto $a$, con derivadas en todos los �rdenes.
  
  El polinomio de primer grado 
\begin{center}
$p_{1}(x) = f(a) + f ' (a) (x-a)$ 
\end{center}
tiene el mismo valor que $f(x)$ en el punto $x=a$ y tambi�n, como se comprueba f�cilmente, la misma derivada que $f(x)$ en este punto. Su gr�fica es una recta tangente a la gr�fica de $f(x)$ en el punto $a$.

  Es posible elegir un polinomio de segundo grado, 
\begin{center}
$p_{2}(x) = f(a) + f ' (a) (x-a) + \frac{1}{2} f '' (a) (x-a) ^ 2$
\end{center}
  tal que en el punto $x=a$ tenga el mismo valor que $f(x)$ y tambi�n valores iguales para su primera y segunda derivada. Su gr�fica en el punto $a$ se acercar� a la de $f(x)$ m�s que la anterior. Es natural esperar que si construimos un polinomio que en $x=a$ tenga las mismas n primeras derivadas que $f(x)$ en el mismo punto, este polinomio se aproximar� m�s a $f(x)$ en los puntos $x$ pr�ximos a $a$. As� obtenemos la siguiente igualdad aproximada, que es la f�rmula de Taylor:

\begin{center} 
$f(x) = f(a) + f '(a) (x-a) + \frac{1}{2!} f '' (a) (x-a) ^ 2 + ...... + \frac{1}{n!} f ^ n(a) (x-a) ^ n$

\end{center}	    
  El segundo miembro de esta f�rmula es un polinomio de grado $n$ en $(x-a)$. Para cada valor de $x$ puede calcularse el valor de este polinomio si se
conocen los valores de $f(a)$ y de sus n primeras derivadas.