%%%%%%%%%%%%%%%%%%%%%%%%%%%%%%%%%%%%%%%%%%%%%%%%%%%%%%%%%%%%%%%%%%%%%%%%%%%%%%%
% Chapter 2: Fundamentos Teóricos
%%%%%%%%%%%%%%%%%%%%%%%%%%%%%%%%%%%%%%%%%%%%%%%%%%%%%%%%%%%%%%%%%%%%%%%%%%%%%%%


\section{Historia}
\label{2:sec:1}
  Brook Taylor ~\cite{URL:XML1} naci\'o el 18 de agosto de 1685 en Edmonton. Hijo de John Taylor,
del Parlamento de Bifrons, Kent, y de Olivia Tempest (hija de Sir Nicholas Tempest).

  En << Los m\'etodos de incrementaci\'on directa e inversa >> de Taylor (1715) agregaba a las
matem\'aticas una nueva rama llamada ahora <<El c\'alculo de las diferencias finitas>> ,
e invent\'o la integraci\'on por partes y descubri\'o la c\'elebre f\'ormula conocida como la Serie de Taylor,
la importancia de esta f\'ormula no fue reconocida hasta 1772, cuando Lagrange proclam\'o los
principios b\'asicos del C\'alculo Diferencial. En su Methodus Incrementorum Directa et Inversa
(Londres, 1715) desarroll\'o una nueva parte dentro de la investigaci\'on matem\'atica,
que hoy se llama c\'alculo de las diferencias finitas. Junto con << Los nuevos principios de
la perspectiva lineal >>. Taylor da cuenta de un experimento para descubrir las leyes de la
atracci\'on magn\'etica (1715) y un m\'etodo no probado para aproximar las ra\'ices de una ecuaci\'on
dando un m\'etodo nuevo para logaritmos computacionales (1717).

  Brook Taylor muri\'o en Somerset House el 29 de diciembre de 1731.

\section{C\'alculo de la serie de Taylor}
\label{2:sec:2}

  Una recta tangente es la mejor aproximaci\'on lineal a la gr\'afica de $f$ en las cercan\'ias del punto de tangencia $(x_{a}, f(x_{a}))$, que es aquella recta que pasa
por el mencionado punto y tiene la misma pendiente que la curva en ese punto \footnote{Primera derivada en el punto}, lo que hace que la recta tangente y la curva sean pr\'acticamente
indistinguibles en las cercan\'ias del punto de tangencia. Si x se encuentra ''lejos'' de $x_{a}$, la recta tangente ya no funciona como aproximador, para esto se tiene que encontrar funci\'on
(no lineal) que cumpla con el prop\'osito.

  La recta tangente es un polinomio de grado 1, el m\'as sencillo tipo de funci\'on que se puede encontrar:

 \begin{center}
$p_{1}(x) = f(a) + f ' (a) (x-a)$

\end{center}
  
tiene el mismo valor que $f(x)$ en el punto $x=a$ y tambi\'en, como se comprueba f\'acilmente, la misma derivada que $f(x)$ en este punto.
 
Es posible elegir un polinomio de segundo grado,
\begin{center}
$p_{2}(x) = f(a) + f ' (a) (x-a) + \frac{1}{2} f '' (a) (x-a) ^ 2$

\end{center}
tal que en el punto $x=a$ tenga el mismo valor que $f(x)$ y tambi\'en valores iguales para su primera y segunda derivada. Su gr\'afica en el punto $a$
se acercar\'a a la de $f(x)$ m\'as que la anterior. Se puede esperar que si se construye un polinomio que en $x=a$ tenga las mismas $n$ primeras derivadas
que $f(x)$ en el mismo punto, este polinomio se aproximar\'a más a $f(x)$ en los puntos $x$ pr\'oximos a $a$. As\'i se obtiene la siguiente igualdad aproximada,
que es la f\'ormula de Taylor:

\begin{center}
$f(x) = f(a) + f '(a) (x-a) + \frac{1}{2!} f '' (a) (x-a) ^ 2 + ...... + \frac{1}{n!} f ^ n(a) (x-a) ^ n$

\end{center}	
  El segundo miembro de esta f\'ormula es un polinomio de grado $n$ en $(x-a)$. Para cada valor de $x$ puede calcularse el valor de este polinomio si se
conocen los valores de $f(a)$ y de sus n primeras derivadas.

  Para el caso de la funci\'on $sin(x)$ el Polinomio de Taylor sería de la siguiente forma:

\begin{center}
$f(x) = sin(a) + cos(a) (x-a) - \frac{1}{2!} sin (a) (x-a) ^ 2 - \frac{1}{3!} cos (a) (x-a) ^3 +$

\end{center}

\begin{center}
$\frac{1}{4!} sin (a) (x-a) ^ 4 + ... + \frac{1}{n!} f ^ n(a) (x-a) ^ n$

\end{center}

\section{F\'ormula de Taylor y de Maclaurin}
\label{2:sec:3}

 Se le denomina f\'ormula de Taylor de $f$ en $a$ a la expresi\'on:
 
\begin{center}
$f(x) = f(a) + f '(a) (x-a) + \frac{1}{2!} f '' (a) (x-a) ^ 2 + ...... + \frac{1}{n!} f ^ n(a) (x-a) ^ n + E_{n}$

\end{center}

Y f\'ormula de Maclaurin de $f$, donde $a = 0$:

\begin{center}
$f(x) = f(0) + f'(0)x + \frac{1}{2!} f^n(0) x^2 + ... + \frac{1}{n!}f^n(0)x^n + E_{n}$

\end{center}

En el caso de $f(x) = sin(x)$ la serie de Maclaurin ser\'ia:

\begin{center}
$sin(x) = x - \frac{x^3}{3!} + \frac{x^5}{5!} - \frac{x^7}{7!} +(-1)^{n+1} \frac {x^{2n+1}}{(2n+1)!} + E_{2n+1}$

\end{center}

Esto ocurre, porque las derivadas de orden par, evaluadas en cero se anulan y las impares se van alternando con valores 1 y -1. A dicha funci\'on se le conoce
como funci\'on impar.

Expresada en notaci\'on sumatoria queda:

\begin{center}

$sen(x)= \sum\limits_{i=1}^{n}(-1)^{n+1} \frac{x^{2n+1}}{(2n+1)!} + E_{2n+1}$

\end{center}








