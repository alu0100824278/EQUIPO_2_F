%%%%%%%%%%%%%%%%%%%%%%%%%%%%%%%%%%%%%%%%%%%%%%%%%%%%%%%%%%%%%%%%%%%%%%%%%%%%%
% Chapter 4: Conclusiones y Trabajos Futuros 
%%%%%%%%%%%%%%%%%%%%%%%%%%%%%%%%%%%%%%%%%%%%%%%%%%%%%%%%%%%%%%%%%%%%%%%%%%%%%%%

 Tras la realizaci\'on de varios c\'odigos en lenguaje Python, los cuales constan de diferentes algoritmos implementados que buscan la soluci\'on al 
planteamiento de dichos c\'odigos, se ha conseguido hallar la aproximaci\'on de f(x)=sin(x) mediante el m\'etodo de Taylor, el error establecido y el tiempo de CPU. 
Para poder analizar y comparar los resultados obtenidos se han llevado a cabo distintos experimentos solicitando en cada uno el grado del polinomio de Taylor,
 el punto central c y el punto x. 

\begin{itemize}
\item
Con dichos valores, se puede afirmar que se debe de aumentar el grado del polinomio tantas veces como sea posible, escoger los puntos x y c con la menor distancia 
existente entre ambos y un mayor n\'umero de cifras decimales, se lograr\'a una mejor aproximaci\'on con un margen menor de error.
En el caso de x = c, el teorema de aproximaci\'on no es adecuado porque solo estaremos calculando la imagen de la funci\'on en el punto c y no una 
aproximaci\'on a la funci\'on.


\item
Al analizar los datos del tiempo de CPU se concluye que al programa Python le toma m\'as tiempo realizar las operaciones con un polinomio de grado 10 que con 
uno de grado 4. Con esto se peude afirmar que cuanto mayor sea el grado del polinomio a calcular m\'as demorar\'a en hacer el c\'alculo, aunque nostros no 
puedamos percibir la diferencia ya que se est\'a trabajando con cent\'esimas y mil\'esimas de segundos. 

\end{itemize}
