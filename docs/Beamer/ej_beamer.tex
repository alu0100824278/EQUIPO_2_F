\documentclass{beamer}
\usepackage[utf8]{inputenc}
\usepackage{graphicx}

\newtheorem{definicion}{Definición}
\newtheorem{ejemplo}{Ejemplo}

%%%%%%%%%%%%%%%%%%%%%%%%%%%%%%%%%%%%%%%%%%%%%%%%%%%%%%%%%%%%%%%%%%%%%%%%%%%%%%%
\title[Series numéricas]{Series numericas: Función sin(x)}
\author[Jorge, Elizabeth, Yessica]{Jorge Antonio Herrera Alonso, Elizabeth Hernández Martín, Yessica Sabrina Gómez Buso}
\date[25-03-2014]{14 de mayo de 2014}
%%%%%%%%%%%%%%%%%%%%%%%%%%%%%%%%%%%%%%%%%%%%%%%%%%%%%%%%%%%%%%%%%%%%%%%%%%%%%%%
\usetheme{Madrid}
%\usetheme{Antibes}
%\usetheme{tree}
%\usetheme{classic}

%%%%%%%%%%%%%%%%%%%%%%%%%%%%%%%%%%%%%%%%%%%%%%%%%%%%%%%%%%%%%%%%%%%%%%%%%%%%%%%
\definecolor{pantone254}{RGB}{122,59,122}
\definecolor{pantone3015}{RGB}{0,88,147}
\definecolor{pantone432}{RGB}{56,61,66}
\setbeamercolor*{palette primary}{use=structure, fg=white,bg=pantone254}
\setbeamercolor*{palette secondary}{use=structure, fg=white,bg=pantone3015}
\setbeamercolor*{palette tertiary}{use=structure, fg=white,bg=pantone432}
%%%%%%%%%%%%%%%%%%%%%%%%%%%%%%%%%%%%%%%%%%%%%%%%%%%%%%%%%%%%%%%%%%%%%%%%%%%%%%%

\begin{document}
  
%++++++++++++++++++++++++++++++++++++++++++++++++++++++++++++++++++++++++++++++  
\begin{frame}

  \titlepage

  \begin{small}
    \begin{center}
     Facultad de Matemáticas \\
     Universidad de La Laguna
    \end{center}
  \end{small}

\end{frame}
%++++++++++++++++++++++++++++++++++++++++++++++++++++++++++++++++++++++++++++++  

%++++++++++++++++++++++++++++++++++++++++++++++++++++++++++++++++++++++++++++++  
\begin{frame}
  \frametitle{Índice}  
  \tableofcontents[pausesections]
\end{frame}
%++++++++++++++++++++++++++++++++++++++++++++++++++++++++++++++++++++++++++++++  

 

\section{Motivación y objetivos}
\begin{frame}
\frametitle{Motivación y bjetivos}
\begin{block}{Motivación}
Aprendizaje orientado al lenguaje de programación $Python$, el procesador de texto $LaTEX$ y el creador de presentaciones $Beamer$.
\end{block}
\begin{block}{Objetivos}
  \begin{itemize}
  \item {\bf Objetivo principal:} Implementación de $Python$ del método de Taylor.\pause
  \item {\bf Objetivo específico:} Aproximación de una función mediante el método de Taylor.
  \end{itemize}
\end{block}


\end{frame}
%++++++++++++++++++++++++++++++++++++++++++++++++++++++++++++++++++++++++++++++  
\section{Fundamentos teóricos}
\begin{frame}

\frametitle{Fundamentos teóricos}

\begin{block}{Definición de una serie de Taylor:}
En matemáticas, una serie de Taylor es una representación de una función como una infinita suma de términos.
\end{block}
\begin{block}{Teorema de Taylor:}

EL Teorema de Taylor permite obtener paroximaciones polinómicas de una función en un entorno de cierto punto en que la dunción sea diferenciables.
Además es el teorema que permite acotar el error obtenido mediante dicha estimación
\end{block}
\begin{block}{Polinomio de Taylor:}
$f(x) = f(a) + f '(a) (x-a) + \frac{1}{2!} f '' (a) (x-a) ^ 2 + ...... + \frac{1}{n!} f ^ n(a) (x-a) ^ n$
\end{block}
	    
\end{frame}


\section{Procedimiento experimental}

\subsection{Descripción de los experimentos}
%++++++++++++++++++++++++++++++++++++++++++++++++++++++++++++++++++++++++++++++  
\begin{frame}
\frametitle{Descripción de los experimentos}

El experimento llevado a cabo en esta memoria ha consistido en la realización de varios códigos en lenguaje $Python$. Los algoritmos implementados que solucionan dichos códigos 
estiman la aproximación $f(x) = sin(x)$ mediante el método de Taylor, solicitando el grado del polinomio de Taylor, el punto central y el punto x donde se evalua dicho polinomio.  
  
\end{frame}
%++++++++++++++++++++++++++++++++++++++++++++++++++++++++++++++++++++++++++++++  

\subsection{Creación de diapositivas}

%++++++++++++++++++++++++++++++++++++++++++++++++++++++++++++++++++++++++++++++  
\begin{frame}
\frametitle{Diapositivas}

\begin{definition}
  Un ejemplo de definición
\end{definition}

\begin{example}
  \begin{itemize}
    \item <1-> Primero \pause
    \item <2-> Segundo \pause
    \item <3-> Tercero \pause
    \item <4-> Cuarto  
  \end{itemize}
\end{example}

\end{frame}
%++++++++++++++++++++++++++++++++++++++++++++++++++++++++++++++++++++++++++++++  

\subsection{Descripción del material:}
%++++++++++++++++++++++++++++++++++++++++++++++++++++++++++++++++++++++++++++++  
\begin{frame}
\frametitle{Descripción del material}

\begin{block}{Materiales:}
  \begin{enumerate}
    \item
      Primero
      \pause

    \item
      Segundo 

  \end{enumerate}
\end{block}

\end{frame}
%++++++++++++++++++++++++++++++++++++++++++++++++++++++++++++++++++++++++++++++  

\section{Bibliografía}
%++++++++++++++++++++++++++++++++++++++++++++++++++++++++++++++++++++++++++++++  
\begin{frame}
  \frametitle{Bibliografía}

  \begin{thebibliography}{10}

    \beamertemplatebookbibitems
    \bibitem[Plan Estudios, 2011]{plan}  
    Documento de verificación del grado. 
    (2011) 

    \beamertemplatebookbibitems
    \bibitem[Guía Docente, 2013]{guia}  
    Guía docente. 
    (2013) 
    {\small $http://eguia.ull.es/matematicas/query.php?codigo=299341201$}

    \beamertemplatebookbibitems
    \bibitem[URL: CTAN]{latex} 
    CTAN. {\small $http://www.ctan.org/$}

  \end{thebibliography}
\end{frame}

%++++++++++++++++++++++++++++++++++++++++++++++++++++++++++++++++++++++++++++++  
\end{document}
